\documentclass[10pt]{article}
\usepackage[margin=0.9in]{geometry}
\usepackage{setspace}
\usepackage{booktabs}
\usepackage{tabularx}
\usepackage{longtable}
\usepackage{graphicx}
\usepackage{float}
\usepackage{hyperref}
\usepackage{amsmath}
\usepackage{siunitx}
\usepackage{natbib}
\usepackage{enumitem}
\usepackage{caption}
\usepackage{subcaption}

\graphicspath{{figures/}}
\hypersetup{colorlinks=true,linkcolor=blue,citecolor=blue,urlcolor=blue}
\setstretch{1.0}
\setlength{\textfloatsep}{8pt plus 2pt minus 2pt}
\setlength{\floatsep}{8pt plus 2pt minus 2pt}

\title{Impact of ERDF on Regional Growth and Convergence in EU NUTS2 Regions (2014--2020):\\Panel Evidence and Policy-Rule Identification}
\author{EUcohesion Project}
\date{\today}

\begin{document}
\maketitle

\begin{abstract}
This report studies the causal effect of ERDF funding intensity on regional GDP per-capita growth and convergence across EU NUTS2 regions. We use the project\'s harmonized panel and policy-rule artifacts already generated in V3.1, combining TWFE benchmarks with policy-rule-based designs (RD and IV). The main outcome is real GDP per-capita growth; PPS growth is reported as robustness. Diagnostics show that the 75\% eligibility cutoff does not generate a strong funding discontinuity (first-stage F-statistics below 2 across bandwidths), so fuzzy RD is not treated as a causal ERDF LATE design. We therefore use diagnostics-driven IV selection among candidate instruments and select a cross-sectional cumulative-exposure specification instrumented by eligibility below 90\%, which delivers the strongest first stage (F=15.65). The headline IV estimates are negative and marginally imprecise (p-values around 0.07--0.09), while TWFE benchmarks remain small and specification-sensitive. Convergence evidence remains strong: both sigma and beta metrics support convergence over the sample. Results should be interpreted with caution because identification still depends on exclusion assumptions and partial eligibility coverage.
\end{abstract}

\section{Introduction}
\textbf{Problematic.} This report targets the causal effect of ERDF intensity on GDP per-capita growth and convergence, with policy period focus on 2014--2020 and post-period outcomes.

The empirical challenge is that regional funding intensity is endogenous to initial conditions and policy targeting. V3.1 therefore combines descriptive panel evidence with explicit design diagnostics for policy-rule identification, rather than presenting one estimator without testing its viability.

\section{Data}
\subsection{Sources and identifiers}
Table~\ref{tab:data-sources} lists the exact datasets used in the repository build and analysis pipeline, with IDs and URLs.
Primary source portals are DG REGIO and Eurostat \citep{dg-regio-data-for-research,eurostat-nama-10r-2gdp,eurostat-nama-10r-2gvagr}, with ERDF finance records from the Cohesion Data Platform \citep{cohesion-99js-gm52,cohesion-tc55-7ysv}.

\begin{table}[H]
\centering
\caption{Data sources used in V3.1}
\label{tab:data-sources}
\small
\begin{tabularx}{\textwidth}{@{}l l l@{}}
\toprule
Domain & Dataset ID & URL \\
\midrule
Cohesion funding & 99js-gm52 & \href{https://cohesiondata.ec.europa.eu/2014-2020-Finances/Finance-Implementation-Details/99js-gm52}{link} \\
Cohesion funding & tc55-7ysv & \href{https://cohesiondata.ec.europa.eu/Historic/Historic-EU-payments-regionalised-and-modelled-/tc55-7ysv}{link} \\
Eurostat GDP (PPS) & nama\_10r\_2gdp & \href{https://ec.europa.eu/eurostat/databrowser/view/nama_10r_2gdp/default/table}{link} \\
Eurostat regional accounts & nama\_10r\_2gvagr & \href{https://ec.europa.eu/eurostat/databrowser/view/nama_10r_2gvagr/default/table}{link} \\
Eurostat population & demo\_r\_d2jan & \href{https://ec.europa.eu/eurostat/databrowser/view/demo_r_d2jan/default/table}{link} \\
Eurostat unemployment & tgs00010 & \href{https://ec.europa.eu/eurostat/databrowser/view/tgs00010/default/table}{link} \\
Eurostat employment & tgs00007 & \href{https://ec.europa.eu/eurostat/databrowser/view/tgs00007/default/table}{link} \\
Eurostat tertiary education & tgs00109 & \href{https://ec.europa.eu/eurostat/databrowser/view/tgs00109/default/table}{link} \\
Eurostat R\&D & tgs00042 & \href{https://ec.europa.eu/eurostat/databrowser/view/tgs00042/default/table}{link} \\
DG REGIO hub & Data for research & \href{https://ec.europa.eu/regional_policy/information-sources/maps/data-for-research_en}{link} \\
\bottomrule
\end{tabularx}
\end{table}

\subsection{Construction summary}
The processed panel is built at the NUTS2-year level with harmonized keys (\texttt{nuts2\_id}, year), ERDF per-capita treatment and lags, GDP outcomes, controls, and eligibility metadata. NUTS harmonization is imperfect across historical revisions, so eligibility/running-variable coverage is partial relative to the full panel.

Real GDP per capita is reconstructed from regional volume indices (\texttt{nama\_10r\_2gvagr}, B1GQ, I15) anchored to each region\'s nominal reference level (preferably 2015):
\[
\text{GDPpc}^{real}_{it} = \text{GDPpc}^{nom}_{i,ref}\times\frac{I^{vol}_{it}}{I^{vol}_{i,ref}}.
\]

Panel coverage metrics from the generated overview are reported in Table~\ref{tab:overview}.

\begin{table}[H]
\centering
\caption{Panel overview metrics}
\label{tab:overview}
\small
\begin{tabular}{ll}
\toprule
metric & value \\
\midrule
Min year & 1986 \\
Max year & 2025 \\
Regions & 398 \\
Rows & 15920 \\
\bottomrule
\end{tabular}

\end{table}

\section{Empirical strategy}
\subsection{TWFE benchmark}
The benchmark panel models are:
\begin{align}
\text{(A)}\quad y_{it} &= \beta\,ERDFpc_{i,t-1} + \gamma'X_{it} + \alpha_i + \tau_t + \varepsilon_{it}, \\
\text{(B)}\quad y_{it} &= \beta\,ERDFpc_{i,t-1} + \gamma'X_{it} + \alpha_i + \delta_{c(i),t} + \varepsilon_{it}, \\
\text{(C)}\quad y_{it} &= \sum_{k=1}^3\beta_k ERDFpc_{i,t-k} + \gamma'X_{it} + \alpha_i + \tau_t + \varepsilon_{it},
\end{align}
where $y_{it}$ is headline real growth (with PPS robustness), $\alpha_i$ are region FE, and errors are clustered by region.

\subsection{RD diagnostics and sharp RD ITT}
Using the running variable
\[
r_i = \left(\frac{GDPpc^{PPS}_i}{EU\ average\ GDPpc^{PPS}}\right)\times 100,
\]
we test whether funding jumps at the 75 threshold. Local linear RD with triangular kernels is estimated over a bandwidth grid $\{5,7.5,10,12.5,15,20\}$.

Because first-stage funding jumps are weak in V3.1, fuzzy RD is treated as non-viable for causal ERDF LATE. Sharp RD is retained as reduced-form eligibility ITT evidence.

\subsection{IV candidate set and selection}
IV candidates are diagnosed via first-stage strength (F-stat, partial $R^2$) before second-stage estimation. Candidate instruments include interactions of eligibility indicators with time exposure and cross-sectional threshold instruments for cumulative ERDF exposure.

Headline IV is selected by the documented rule: choose the strongest candidate with first-stage F$\geq$10 for the headline outcome window; otherwise revert to conservative TWFE interpretation.

\section{Results}
\subsection{TWFE benchmark estimates}
Table~\ref{tab:twfe} reports Models A and B for real growth; Table~\ref{tab:dl} reports distributed lags. Figure~\ref{fig:dynamic} visualizes dynamic lag estimates.

\begin{table}[H]
\centering
\caption{TWFE benchmark (headline outcome: real growth)}
\label{tab:twfe}
\small
\begin{tabular}{lllllll}
\toprule
Model & Coef & SE & p & N obs & N regions & SE cluster \\
\midrule
Model A & 0.0057 & 0.0030 & 0.0604 & 1175 & 223 & nuts2 \\
Model B & 0.0034 & 0.0029 & 0.2444 & 1175 & 223 & nuts2 \\
\bottomrule
\end{tabular}

\end{table}

\begin{table}[H]
\centering
\caption{Distributed lags (Model C, headline outcome)}
\label{tab:dl}
\small
\begin{tabular}{lllll}
\toprule
Term & Coef & SE & p & N obs \\
\midrule
ERDF pc lag 1 & 0.0120 & 0.0048 & 0.0134 & 861 \\
ERDF pc lag 2 & 0.0079 & 0.0038 & 0.0366 & 861 \\
ERDF pc lag 3 & -0.0031 & 0.0027 & 0.2470 & 861 \\
\bottomrule
\end{tabular}

\end{table}

\begin{figure}[H]
\centering
\begin{subfigure}{0.49\textwidth}
\centering
\includegraphics[width=\linewidth]{dynamic_lag_response_v31.png}
\caption{Dynamic lag response}
\end{subfigure}
\begin{subfigure}{0.49\textwidth}
\centering
\includegraphics[width=\linewidth]{sigma_convergence_v31.png}
\caption{Sigma convergence}
\end{subfigure}
\caption{Dynamic treatment profile and convergence trend}
\label{fig:dynamic}
\end{figure}

\subsection{RD diagnostics and sharp RD}
Table~\ref{tab:rd-fs} and Figure~\ref{fig:rd-fs} show weak funding discontinuity at 75 (F-statistics below conventional strength thresholds). Table~\ref{tab:rd-out} reports sharp-RD ITT estimates; Table~\ref{tab:rd-placebo} reports placebo pre-trend.

\begin{table}[H]
\centering
\caption{RD first-stage funding jump diagnostics at 75 cutoff}
\label{tab:rd-fs}
\small
\begin{tabular}{llllll}
\toprule
BW & Jump & SE & p & F & Fuzzy viable \\
\midrule
5.000 & 605.066 & 582.900 & 0.299 & 1.077 & False \\
7.500 & 628.144 & 482.716 & 0.193 & 1.693 & False \\
10.000 & 521.676 & 405.835 & 0.199 & 1.652 & False \\
12.500 & 406.564 & 356.718 & 0.254 & 1.299 & False \\
15.000 & 323.631 & 308.232 & 0.294 & 1.102 & False \\
20.000 & 277.562 & 251.487 & 0.270 & 1.218 & False \\
\bottomrule
\end{tabular}

\end{table}

\begin{figure}[H]
\centering
\begin{subfigure}{0.49\textwidth}
\centering
\includegraphics[width=\linewidth]{rd_first_stage_funding_jump_v31.png}
\caption{Funding jump at cutoff}
\end{subfigure}
\begin{subfigure}{0.49\textwidth}
\centering
\includegraphics[width=\linewidth]{rd_outcome_binned_scatter_v31.png}
\caption{Outcome sharp RD}
\end{subfigure}
\caption{RD diagnostics and reduced-form eligibility discontinuity}
\label{fig:rd-fs}
\end{figure}

\begin{table}[H]
\centering
\caption{Sharp RD outcome discontinuities (eligibility ITT)}
\label{tab:rd-out}
\small
\begin{tabular}{lllllll}
\toprule
Outcome & Window & BW & Coef & SE & p & N \\
\midrule
PPS growth & post\_2016\_2020 & 10.000 & -1.971 & 1.536 & 0.200 & 53 \\
PPS growth & post\_2021\_2023 & 10.000 & 2.840 & 1.202 & 0.018 & 53 \\
Real growth & post\_2016\_2020 & 10.000 & -0.159 & 1.375 & 0.908 & 53 \\
Real growth & post\_2021\_2023 & 10.000 & 2.420 & 0.975 & 0.013 & 53 \\
\bottomrule
\end{tabular}

\end{table}

\begin{table}[H]
\centering
\caption{RD placebo pre-trend test}
\label{tab:rd-placebo}
\small
\begin{tabular}{llllll}
\toprule
Window & BW & Coef & SE & p & N \\
\midrule
pre\_2010\_2013 & 10.000 & -1.942 & 2.456 & 0.429 & 53 \\
\bottomrule
\end{tabular}

\end{table}

\subsection{IV diagnostics and headline IV}
Table~\ref{tab:iv-candidates} summarizes first-stage diagnostics across candidates. The selected headline design is cross-sectional 2SLS using cumulative exposure instrumented by \texttt{eligible\_lt90}. Table~\ref{tab:iv-cross} reports those estimates; panel IV is reported in Table~\ref{tab:iv-panel}.

\begin{table}[H]
\centering
\caption{IV first-stage candidate diagnostics}
\label{tab:iv-candidates}
\tiny
\begin{tabular}{llllllll}
\toprule
ID & Sample & Endogenous & Instrument & F & Partial R2 & F (headline) & Status \\
\midrule
Z1 & panel & erdf\_eur\_pc\_l1 & z1\_post75 &  &  &  & failed \\
Z4\_75 & cross\_section & erdf\_eur\_pc\_cum\_2014\_2020 & eligible\_lt75 & 10.71 & 0.04 & 4.60 & ok \\
Z4\_75 & cross\_section & erdf\_eur\_pc\_cum\_2015\_2020 & eligible\_lt75 & 10.71 & 0.04 & 4.60 & ok \\
Z4\_90 & cross\_section & erdf\_eur\_pc\_cum\_2014\_2020 & eligible\_lt90 & 9.61 & 0.04 & 15.65 & ok \\
Z4\_90 & cross\_section & erdf\_eur\_pc\_cum\_2015\_2020 & eligible\_lt90 & 9.61 & 0.04 & 15.65 & ok \\
Z5 & panel & erdf\_eur\_pc\_l1 & z5\_eu90 & 1.44 & 0.00 &  & ok \\
Z2 & panel & erdf\_eur\_pc\_l1 & z2\_eu75 & 1.23 & 0.00 &  & ok \\
Z3 & panel & erdf\_eur\_pc\_l1 & z3\_country75 & 1.12 & 0.00 &  & ok \\
\bottomrule
\end{tabular}

\end{table}

\begin{table}[H]
\centering
\caption{Cross-sectional 2SLS (headline IV family)}
\label{tab:iv-cross}
\small
\begin{tabular}{llllllll}
\toprule
Outcome & Window & Instrument & Coef & SE & p & F & N \\
\midrule
Real growth & post\_2016\_2020 & eligible\_lt90 & -0.004 & 0.002 & 0.088 & 15.654 & 265 \\
Real growth & post\_2021\_2023 & eligible\_lt90 & -0.005 & 0.003 & 0.069 & 15.654 & 265 \\
PPS growth & post\_2016\_2020 & eligible\_lt90 & -0.002 & 0.002 & 0.140 & 15.654 & 265 \\
PPS growth & post\_2021\_2023 & eligible\_lt90 & -0.001 & 0.002 & 0.651 & 15.654 & 265 \\
\bottomrule
\end{tabular}

\end{table}

\begin{table}[H]
\centering
\caption{Panel 2SLS (selected panel candidate)}
\label{tab:iv-panel}
\small
\begin{tabular}{lllllll}
\toprule
Outcome & Instrument & Coef & SE & p & F & N \\
\midrule
Real growth & z5\_eu90 & -0.087 & 0.095 & 0.360 & 1.810 & 1175 \\
\bottomrule
\end{tabular}

\end{table}

\begin{figure}[H]
\centering
\includegraphics[width=0.62\textwidth]{iv_first_stage_scatter_v31.png}
\caption{IV first stage: fitted vs observed cumulative exposure}
\label{fig:iv-fs}
\end{figure}

\subsection{Comparison and convergence}
Table~\ref{tab:comparison} compiles the main estimator families. Table~\ref{tab:beta} summarizes beta convergence evidence.

\begin{table}[H]
\centering
\caption{Model comparison summary (V3.1)}
\label{tab:comparison}
\footnotesize
\begin{tabular}{llllllll}
\toprule
Family & Model & Window & Coef & SE & p & F & Headline \\
\midrule
IV (cross-section) & IV 2SLS cross-section (selected candidate) & post\_2016\_2020 & -0.004 & 0.002 & 0.088 & 15.654 & True \\
IV (panel) & IV 2SLS panel (selected candidate) & panel & -0.087 & 0.095 & 0.360 & 1.810 & False \\
RD (sharp ITT) & Sharp RD bw=10.0 & post\_2016\_2020 & -0.159 & 1.375 & 0.908 &  & False \\
TWFE & Model A & panel & 0.006 & 0.003 & 0.060 &  & False \\
TWFE & Model B & panel & 0.003 & 0.003 & 0.244 &  & False \\
TWFE-dynamic & Model C (l1 term) & panel & 0.012 & 0.005 & 0.013 &  & False \\
\bottomrule
\end{tabular}

\end{table}

\begin{table}[H]
\centering
\caption{Beta convergence (Model D)}
\label{tab:beta}
\small
\begin{tabular}{lllll}
\toprule
Outcome & Term & Coef & SE & p \\
\midrule
Real growth & log real GDP pc lag & -5.267 & 0.602 & 0.000 \\
PPS growth & log PPS GDP pc lag & -7.725 & 0.436 & 0.000 \\
\bottomrule
\end{tabular}

\end{table}

\section{Discussion and limitations}
\begin{itemize}[leftmargin=1.2em]
    \item \textbf{Weak fuzzy RD:} at the 75 cutoff, first-stage funding-jump F-statistics remain below 2 across bandwidths, so fuzzy RD is not credible for ERDF LATE.
    \item \textbf{IV assumptions:} the headline cross-sectional IV depends on exclusion and monotonicity assumptions tied to policy thresholds; estimates should be interpreted as design-based but assumption-dependent.
    \item \textbf{Coverage mismatch:} eligibility/running-variable mapping does not cover all region-year observations in the full panel because of NUTS revision overlap.
    \item \textbf{Outcome construction:} real GDP per capita is reconstructed from volume indices anchored to nominal levels, not direct chain-linked euro series.
\end{itemize}

\section{Conclusion}
With the currently available policy structure and data, V3.1 supports a cautious interpretation: convergence patterns are robustly negative in beta terms and declining in sigma terms, while direct ERDF growth effects remain sensitive to estimator choice. RD funding-jump diagnostics indicate weak support for fuzzy RD identification at the 75 threshold. The strongest available design in this pipeline is cross-sectional IV with cumulative exposure instrumented by \texttt{eligible\_lt90}, which achieves a stronger first stage but still yields imprecise headline effects. The defensible claim is therefore limited: evidence is consistent with non-zero policy impacts in some specifications, but large causal claims are not yet supported without stronger quasi-experimental variation.

\bibliographystyle{plainnat}
\bibliography{references}

\end{document}
